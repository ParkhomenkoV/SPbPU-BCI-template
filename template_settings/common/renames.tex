%%% Переопределение именований %%%
\renewcommand{\alsoname}{см. также}
\renewcommand{\seename}{см.}
\renewcommand{\headtoname}{вх.}
\renewcommand{\ccname}{исх.}
\renewcommand{\enclname}{вкл.}
\renewcommand{\pagename}{Pages}
\renewcommand{\partname}{Часть}
\renewcommand{\abstractname}{\textbf{Аннотация}}
\newcommand{\abstractnameENG}{\textbf{Annotation}}
\newcommand{\keywords}{\textbf{Ключевые слова}}
\newcommand{\keywordsENG}{\textbf{Keywords}}
\newcommand{\acknowledgements}{\textbf{Благодарности}}
\newcommand{\acknowledgementsENG}{\textbf{Acknowledgements}}
\renewcommand{\contentsname}{Content} % 
%\renewcommand{\contentsname}{Содержание} % (ГОСТ Р 7.0.11-2011, 4)
%\renewcommand{\contentsname}{Оглавление} % (ГОСТ Р 7.0.11-2011, 4)
\renewcommand{\figurename}{Рис.} % Стиль СПбПУ
%\renewcommand{\figurename}{Рисунок} % (ГОСТ Р 7.0.11-2011, 5.3.9)
\renewcommand{\tablename}{Таблица} % (ГОСТ Р 7.0.11-2011, 5.3.10)
%\renewcommand{\indexname}{Предметный указатель}
\renewcommand{\listfigurename}{Список рисунков}
\renewcommand{\listtablename}{Список таблиц}
\renewcommand{\refname}{\fullbibtitle}
\renewcommand{\bibname}{\fullbibtitle}

\newcommand{\chapterEnTitle}{Сhapter title} % <- input the English title here (only once!) 
\newcommand{\chapterRuTitle}{Название главы}          % <- введите 
\newcommand{\sectionEnTitle}{Section title} %<- input subparagraph title in english
\newcommand{\sectionRuTitle}{Название подраздела} % <- введите название подраздела по-русски
\newcommand{\subsectionEnTitle}{Subsection title} % - input subsection title in english
\newcommand{\subsectionRuTitle}{Название параграфа} % <- введите название параграфа по-русски
\newcommand{\subsubsectionEnTitle}{Subsubsection title} % <- input subparagraph title in english
\newcommand{\subsubsectionRuTitle}{Название подпараграфа} % <- введите название подпараграфа по-русски
