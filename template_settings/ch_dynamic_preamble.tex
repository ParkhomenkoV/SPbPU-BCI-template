%%% Предметный указатель  ГОСТ 7.78-99 Index %%%
%c обобщенными рубриками или развернутый
%или указатель терминов (в общем случае - произвольное число указателей)
%подключать до hyperref

%\usepackage{makeidx} %возможно, необходимо подключить И/ИЛИ пройти Tools-> Commands -> MakeIndex

\usepackage{imakeidx} 
%\indexsetup{level=\section*,toclevel=section,noclearpage}
\makeindex[program=makeindex,
options=-s template_settings/common/myindex.ist, %подключаем стилевой файл для форматирования вывода
name=ru, % префикс для русских указателей 
% если убрать <<ru>>, то для работы дефолтового придется вручную включать Tools-> Commands -> MakeIndex
title={\chapterLight{} 
%   \hrule{}
	Предметный указатель
%	\hrule{}
} 
%,columns=1 %по умолчанию 2
]
\makeindex[program=makeindex,
options=-s template_settings/common/myindex.ist, %подключаем стилевой файл для форматирования вывода
name=en, % префикс для английских указателей
title={\chapterLight{}
%	\hrule{}
	Index
%	\hrule{}
} 
%,columns=1 %по умолчанию 2
] 
%убрать добавление <<title>> в содержание:
%\noindexintoc %not to add index title in PURE makeidx %intoc is false by default with imakeidx

%%%%%%%%%%%%%%%%%%%%%%%%%%%%%%%%%%%%%%%%%%%%%%%%%%%%%%%%%%%%%%%%%%%%%%%%%%%%%%%%%%%%%%%%%%%%%%%%%%%%%%
%%%%%%%%%%%%%% FROM DISERTATION %%%%%%%%%%%%%%%%%%%%%
%%% Алгоритмы %%%

%\usepackage[linesnumbered]{algorithm2e}
\usepackage[linesnumbered,vlined,figure,scleft]{algorithm2e}

%% Glogal params %%
%ruled, tworuled, algoruled --- put lines to wrap the caption plus a line at the bottom (top) - one should not use this together with inline captions!
%vlined 										--- instead of begin...end will be lines
%boxed 											--- make a box
% figure 										--- count as Fig. ...


% Settings of caption       --- if one will use \caption{} option 	instead of inline + environment caption.
%\SetAlgoCaptionSeparator{.}
%\SetAlgorithmName{Algorithm}{} %last arg is the title of listing table


% Settings for lines numbers
\SetNlSkip{0em}							% sets the value of the space between the line numbers and the text, by default 1em.
\SetNlSty{normalsize}{\hphantom{0}}{.}%defines how to print line numbers
%\hspace*{5mm} does not work 
\SetAlgoNlRelativeSize{-1}	% sets the relative size of line numbers. By default, line numbers are two size smaller than algorithm text

% How to ignore line nuber and to wrap
%http://tex.stackexchange.com/questions/153646/algorithm2e-disabling-line-numbers-for-specific-lines
%http://tex.stackexchange.com/questions/86580/how-to-adjust-line-numbers-of-algorithm2e-package
\makeatletter
%\newcommand{\nosemic}{\renewcommand{\@endalgocfline}{\relax}}% Drop semi-colon ;
%\newcommand{\dosemic}{\renewcommand{\@endalgocfline}{\algocf@endline}}% Reinstate semi-colon ;
%\newcommand{\pushline}{\Indp}% Indent
%\newcommand{\popline}{\Indm\dosemic}% Undent
\let\oldnl\nl% Store \nl in \oldnl
\newcommand{\nonl}{\renewcommand{\nl}{\let\nl\oldnl}}% Remove line number for one line
\makeatother


% Settings for vlines 			
%\SetInd{0.3em}{0.5em}			%default and spaces before and after are 0.5em and 1em
%\SetVlineSkip{5em}					% Sets the value of the vertical space after the little horizontal line which closes a block in vlined mode

%% User abbreviations for ASTRA %%
\SetKwInput{KwInput}{Input}
\SetKwInput{KwOutput}{Output}
%% See also %%
%http://tex.stackexchange.com/questions/145114/caption-below-boxed-algorithm2e-when-used-as-a-figure
%http://tex.stackexchange.com/questions/83536/align-comments-in-algorithm-with-package-algorithm2e

